\documentclass{article}
%Russian-specific packages
\usepackage[utf8]{inputenc}
\usepackage[english,russian]{babel}
\usepackage{ amssymb }
\usepackage{ upgreek }
\usepackage{ graphicx }
\usepackage{indentfirst}
\usepackage[left=20mm, top=20mm, right=20mm, bottom=20mm, nohead, nofoot]{geometry}
\usepackage{ textcomp }
\usepackage{ setspace }
\usepackage{ float }

\begin{document}
\title{Определение вязкости воздуха по скорости истечения через капилляр.}
\author{Лебедева Ирина, Б01-902}
\date{26 апреля 2020 года}

\maketitle

\section{Цель работы}
Измерение вязкости $\eta$ воздуха по измерению объёма газа, протекающего через капилляр (иглу шприца) при переменномперепаде давления. Проверка зависимости расхода газа $Q$ от радиуса капилляра $r$:
$Q$ $\sim$ $r^4$ (формула Пуазейля).
\section{Оборудование}
Шприц на 20 мл без поршня, сменные капилляры разныхдиаметров $d$ и длин $l$, секундомерс возможностью фиксации промежуточных значений,прозрачный цилиндрический стакансводой, линейка, небольшой кусочек ластика.
\section{Введение}

\

При небольших скоростях газа или жидкоститечение среды является ламинарным. С увеличением скорости потока движение приобретает сложный, запутанный характер, слои перемешиваются, течение становится турбулентным.
\noindentХарактер движения газа или жидкости зависит от соотношения между кинетической энергией движущейся среды и работой сил вязкости. Отношение плотности кинетической энергии $\sim$ $pv^2$ к плотности энергетических потерь из-за работы сил вязкости  $\sim$ $\eta v/a$ определяет безразмерное число Рейнольдса:
\begin{equation}
\label{eq:fourierrow}{Re} = \frac{vr\rho}{\eta},
\end{equation}
где $v$ - характерная скорость течения, $\eta$ - вязкость жидкости или газа, $\rho$ - плотность среды, $r$  - некоторый характерный размер задачи.

\

В гладких трубах круглого сечения переход от ламинарного течения к турбулентному происходит при значениях $Re$ $\sim$ 1000.

\

Закон изменения уровня воды в цилиндре с течением времени: 
\begin{equation}
\label{eq:fourierrow}{t} = {\beta}ln\frac{h_0}{h},
\end{equation}
где $h_0$ - начальное значение уровня воды в цилиндре, $h$ - значение уровня воды в цилинтре в момент времени $t$, {\beta} = \frac{8Sl\eta}{pighr^4\rho_в}.

\

Таким образом мы сможем определить вязкость воздуха по формуле:
\begin{equation}
\label{eq:fourierrow}{\eta} = \frac{\pi gr^4\rho\beta}{8lS},
\end{equation}
\newpage
\section{Экспериментальная установка}
\begin{figure}[h]
\centering
\includegraphics[width=0.53\linewidth]{sd.png}
\end{figure}

\



\section{Отчет по выполнению работы}
\begin{enumerate}
\item Определим параметры всех имеющихся в наборе сменных капилляров: 

\

{
\centering
\begin{center}
\begin{tabular}{c c}
$l_1$ = 30 мм & $d_1$ = 0,4 мм\\
$l_2$ = 40 мм & $d_2$ = 0,5 мм\\
$l_3$ = 40 мм & $d_3$ = 0,6 мм\\
\end{tabular}
\end{center}
}
\item Рассчитаем площадь $S$ внутреннего поперечного сечения цилиндра
\begin{center}
    $S$ = 3,14 $cm^2$
\end{center}
\newpage
\item Снимем зависимость t(h) времени заполнения цилиндра от нижней метки $h_0$ = 20 мл до уровня $h$ :
\begin{table}[H]
\begin{center}{
\caption{\label{tab:canonsummary}Зависимость времени заполнения цилиндра от уровня нахождения жидкости.}

\

\centering
\begin{tabular}{|c|c|c|c|c|c|c|c|c|c|c|c|c|c|c|c|c|c|}
\hline
$h,$ мм & 69,0 & 65,5 & 62,0 & 58,5 & 55,0 & 51,5 & 48,0 & 44,5 & 41,0 & 37,5 & 34 & 30,5 & 27,0 & 23,5 & 20,0 & 16,5 & 13,0\\
\hline
$t,$ с & 0 & 2 & 4 & 6 & 8 & 10 & 13 & 15 & 18 & 21 & 24 & 28 & 32 & 37 & 43 & 49 & 57 \\
\hline
\end{tabular}
}
\end{center}
\end{table}
\item Снимем зависимость t(d) времени заполнения цилиндра между двумя фиксированными уровнями от диаметра капилляра $d$. 
\begin{table}[H]
\begin{center}{
\caption{\label{tab:canonsummary}Зависимость времени заполнения цилиндра от диаметра капилляра.}

\

\centering
\begin{tabular}{|c|c|}
\hline
$d,$ мм & $t,$ с \\
\hline
0,4 & 24,26  \\
\hline
0,5 & 12,83 \\
\hline
0,6 & 7,1 \\
\hline
\end{tabular}
}
\end{center}
\end{table}
\newpage
\item Построим график зависимости t($ln\frac{h_0}{h}$), используя данные из пункта 3.
\begin{figure}[h]
\centering
\includegraphics[width=0.53\linewidth]{vis.png}
\caption{График зависимости t($ln\frac{h_0}{h}$)}
\end{figure}

\

Уравнение прямой y = 34x + 0,17. Погрешность рассчета $\beta$ по МНК $\upvarepsilon_\beta$ = 5,6\%

\

По наклону прямой определим коэффициент $\beta$ = 34 и с помощью формулы (3) найдем вязкость воздуха: \begin{center}
    $\eta$ = 11,6$\cdot$$10^-^6$ Па$\cdot$c
\end{center}
Погрешность  $\upvarepsilon_\eta$ = \sqrt{\upvarepsilon^2_\beta + \upvarepsilon^2_l + \upvarepsilon^2_S} = 7,3 \%

\newpage
\item Построим график зависимости t(1/$d^4$),  используя данные из пункта 4.
\begin{figure}[h]
\centering
\includegraphics[width=0.50\linewidth]{vis1.png}
\caption{График зависимости t(1/$d^4$)}
\end{figure}

\

Уравнение прямой y = 0,50x + 4,9. Погрешность рассчета $\alpha$ по МНК $\upvarepsilon_\alpha$ = 12,87\%
По наклону прямой определим коэффициент $\alpha$ = 0,50 и с помощью формулы (3) найдем вязкость воздуха: \begin{center}
    $\eta$ = 15,3$\cdot$$10^-^6$ Па$\cdot$c
\end{center}
Погрешность  $\upvarepsilon_\eta$ = \sqrt{\upvarepsilon^2_\alpha + \upvarepsilon^2_l + \upvarepsilon^2_S} = 13,8 \%

\

\item Вычислим $Re$, используя формулу (1). Учтем, что {v} = $\frac{\pi r^4 0,5\rho g h_0}{8lS\eta} = 0,0015 M/C,$

\

\begin{center}
    $Re_\beta$ = 51,7 \\

    $Re_\alpha$ = 39,2

\end{center}
\item Вычислим длину установления ламинарного течения:

\begin{center}
    $l_\beta$ = 0,2r$Re_\beta$ = 4,14 мм, \\
    $l_\alpha$ = 0,2r$Re_\alpha$ = 3,14 мм
\end{center}
\end{enumerate}
\section{Вывод} 
С помощью двух экспериментов измерили вязкость воздуха: $\eta_1$ = 11,6\cdot$10^-6$ Па\cdot c , \ $\eta_2$ = 15,3\cdot$10^-6$ Па\cdot c. $ Сравнивая результат экспериментов с табличным значением вязкости воздуха $\eta$ = 17,8\cdot$10^-6$ Па\cdot c$ замечаем, что измерение вязкости вторым способом является более верным, так как результат второго эксперимента ближе к табличному. Большое отклонение первого результата от табличного значения может быть вызвано условиями проведения эксперимента, так как установка была собрана в домашних условиях.
\end{document}
